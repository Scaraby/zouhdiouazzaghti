\documentclass[12pt,a4paper]{report}
\usepackage{graphicx}
\usepackage[frenchb]{babel}
\addto\captionsfrench{\renewcommand{\chaptername}{Chapitre}}
\usepackage[utf8]{inputenc}
\usepackage{caption}
\captionsetup[figure]{labelformat=empty}
\begin{document}
\begin{titlepage}


	\centering
	\includegraphics[width=0.20\textwidth]{ensea.png}\par\vspace{1cm}
	{\scshape\LARGE Ecole Nationale Supérieure de l'Electronique et de ses Applications \par}
	\vspace{1cm}
	{\scshape\Large Projet Latéral Transversal\par}
	\vspace{1.5cm}
	{\huge\bfseries Tales of Kornwal\par}
	\vspace{2cm}
	{\Large\itshape OUAZZAGHTI Reda\par et\par ZOUHDI Zakaria\par}
	\vfill
	Projet de troisième année supervisé par\par
	M.~\textsc{Granier} et Prof.~\textsc{Gosselin}

	\vfill

% Bottom of the page
	{\large 29 Septembre 2016\par \tiny \begin{flushleft}
	Version 1.1 du 29/09/16 \today
\end{flushleft}	 }

\end{titlepage}

\tableofcontents
    \chapter{Objectif}
    \section{Présentation générale}
    \begin{figure*}[htp]
    \includegraphics[width=0.80\textwidth]{bridge.png}
    \hfill
  \caption {\tiny By David Revoy / Blender Foundation - Own work, CC BY 3.0}
\end{figure*}
    Tales of Kornwal est un jeu vidéo basé sur les mêmes règles de Fallout Tactics, \textit{i.e}  un jeu d'aventure doté d'un système de combat en tour par tour, permettant aux joueur de progresser et interagir avec un univers à l'allure originale à mi-chemin entre le médiéval-fantastique et le post-apocalyptique.
    
    Les interactions seront basées sur un système de gestion d'inventaire et d'au moins une caractéristique, qui feront office de modificateurs lors d'actions enterprises par le personnage (\textit{e.g} : la caractéristique "Force" influera grandement sur les dégâts infligés par un ennemi ou par le héros, ainsi que l'utilisation de telle ou telle arme).
    
    \section{Règles du jeu}
    Le jeu pourra posséder plusieurs aspects dépendant de l'étude du cahier des charges :
    \par\leavevmode\
\begin{itemize}
\item Déplacement d'un personnage sur une "zone" de la mappemonde, accédant aux différentes cases nord-sud-est-ouest de la map en cliquant sur l'une des extrémités de l'écran.
\par\leavevmode\

\item Système de combat tour par tour : lors d'une rencontre avec un ennemi, la map se vide de tous les sprites autres que le personnage joueur et ses adversaires, laissant donc place au duel entre le héros et l'ennemi. Le joueur commencera en premier (sauf modification) et disposera de deux choix possibles : se déplacer d'une case dans la zone, ou attaquer l'ennemi, faisant baisser son capital de points de vie. Le nombre de points de vie retirés dépendra de la caractéristique FORCE du personnage, ainsi que de son ARME équipée. Le tour se termine après que l'une des actions suivante a été effectuée, laissant le tour à l'ennemi (qui se déroulera de la même manière).






\par\leavevmode\

\item Système d'interaction avec les personnages joueur ou non-joueur (discussions, interface d'échange d'objets)



\end{itemize}
    
    
\end{document}
